
\documentstyle[11pt]{article}
\textheight 9.0in
\textwidth 6.5in
\oddsidemargin 0.0in
\topmargin -0.5in
\setlength{\floatsep}{0.3cm}
\setlength{\textfloatsep}{0.3cm}
\setlength{\intextsep}{0.3cm}
\newcommand{\definedas} {\stackrel{\Delta} {=}}
% Include the macros for including postscript figures in latex - CSG
\input{./psfig}


%\parindent 0pt
\begin{document}
% \setlength{\parindent}{5pt}
% \setlength{\parskip}{0.25cm}
\bibliographystyle{unsrt}

\begin{titlepage}
\vspace*{0.625in}

\begin{center}
{\Large \bf 
An Excerpt From: \\
\vskip 1.0cm
`Partitioning Digital Circuits   \\
\vskip 0.25cm
for Implementation in Multiple FPGA ICs' \\
}
\vspace{0.75cm}
{\normalsize \bf Roman Ku\v znar$^{1,2}$, Franc Brglez$^{1}$, Krzysztof
Kozminski$^{1}$} 
\vskip 1.0cm
\end{center}
\vspace{0.75in}
\footnotesize{

\indent The following pages contain an excerpt from the MCNC Technical Report
named above. This information is provided in order to facilitate the use of
the Partitioning (Xilinx) benchmarks developed as part of the project
described in this MCNC Technical Report.

\vspace{0.15in}
Memoranda in this series are issued for early dissemination to MCNC,
its six Participating Institutions, and MCNC Industrial Affiliates.
They are not considered to be published merely by virtue of appearing
in this series.  This copy is for private circulation only and may not
be further copied or distributed.

\vspace{0.15in}
MCNC Technical Reports are routinely submitted for publication
elsewhere.  Their distribution outside the MCNC Community prior to
publication is limited to peer communications and specific requests.
References to this work should be made either to the published version,
if any, or in the form ``private communication.''

\vspace{0.15in}
For information about the ideas expressed herein, contact the author(s)
directly.  For information about the MCNC Technical Report Series, or
Industrial Affiliates Program, contact Corporate Communications, MCNC,
P.O. Box 12889, Research Triangle Park, NC 27709; 919-248-1842.

\begin{center}

\vspace{0.25in}
\today

Technical Report TR93-03	
\vskip 0.3cm 
\copyright 1993 MCNC
All Rights Reserved
\end{center}
}

\vspace{0.875in}
\vspace{0.1in}
\footnoterule
\vskip 0.1in  
{\footnotesize 
\noindent
        $^{1}$MCNC Center for Microelectronic Systems Technologies,
        P.O. Box 12889,
        Research Triangle Park, NC 27709 \\

\noindent $^{2}$ Department of Electrical and Computer Engineering,
University of Ljubljana,
                 Tr\v za\v ska 25, 61000 Ljubljana, Slovenia
}
\end{titlepage}



%Cover_abstract.tex

\vskip 5.75cm

\begin{center}

{\Large \bf
Partitioning Digital Circuits   \\
\vskip 0.25cm
for Implementation in Multiple FPGA ICs \\
}
\vspace{0.75cm}
{\normalsize \bf Roman Ku\v znar, Franc Brglez, Krzysztof
Kozminski}

\vskip 3.0cm

{\large \bf ABSTRACT}
\end{center}
\vskip 0.5cm

This report considers the problem of obtaining a minimum--cost
partitioning of a large logic circuit into a collection of 
subcircuits such that each subcircuit will fit into some
device selected from a given library.
Each device in the library may have a different price, size, and
terminal capacity associated with it.
This work has been motivated by a need to provide digital system designers
with a capability to implement large circuits that do not fit in any one
of the commercially available FPGA devices.
Automatic partitioning software is needed to relieve the designers from the
tedious task of partitioning the design manually in such situations.

\vskip 6pt

The main result of the performed research is a partitioning algorithm
based on a recursive application of the Fiduccia-Mattheyses bipartitioning
heuristic.
This heuristic has been amended
with appropriate extensions to handle (a) the overall goal of the
cost minimization and (b) the constraints reflecting the
limitations on the capacity of FPGA chips.
While the formulation and the implementation 
are general, we demonstrate the application of our method
using circuits generated with a commercial tool and partitioned
for implementation with a specific library of Xilinx's FPGA devices.

\vskip 6pt

The experimental partitioning software, {\sl k-way.x}, developed in the
course of the reported work, has exhibited a very encouraging performance,
being able to produce solutions with cost quite near to the theoretical
minimum calculated for many of the benchmark circuits.




\section{Comments}
The following section contains a description of how the
{\em k--way.x} \cite{Kuznar93a} partitioning algorithm was interfaced to
the Xilinx tool, XACT. Other partitioning algorithms may be interfaced with
the Xilinx Design Development System in a similar manner. This document is
intended to provide users of the Partitioning benchmark suite with guidelines
about interfacing their partitioning algorithms to the Xilinx Design
Development System. 

\section{Software Implementation of the k--way Partitioning Algorithm}

We implemented the proposed k--way partitioning algorithm in the C 
language and designed interfaces to the Xilinx tool XACT \cite{Xil91a}.
In Figure \ref{k-way.interface} we illustrate the process of using our 
partitioning tool, {\sl k--way.x}, as an extension of the XACT Design 
Development System.

\subsection{Interface to the XACT Xilinx Development System}

\begin{figure}[htb]
\centerline{
\psfig{figure=fig.1,prolog=mac.pro}
}
%\vspace*{0.5in}
%\vspace*{6.2in}
\caption{Interfacing $k-way.x$ to XACT}
\label{k-way.interface}
\end{figure}



The input data to {\sl k--way.x} is a design description given in the 
$XNF$ format as a netlist of gate primitives.
Using the XACT software, this description can be readily mapped into a 
netlist of CLBs and IOBs represented in the $MAP$ format.
The need for partitioning arises when the mapped netlist does not fit 
into any FPGA package available in the device library.
Furthermore, partitioning can be used to minimize the design cost, even 
if it is possible to implement the design as a single, large FPGA IC.
For example, it is possible that the smallest FPGA chip with a 
sufficient number of CLBs is unroutable, forcing the designer to use
a larger device, capable of not only containing all CLBs but also 
yielding a 100\% routing completion.
This larger device may be significantly oversized compared to the 
design size; in other words, its CLB utilization may be very poor.
If the design is partitioned, it is possible that a number of smaller 
devices with higher CLB utilization could be used, thus reducing the 
design cost.
\par
Currently, the designer has to perform the partitioning manually.
In certain cases this task may be manageable, e.g., if the circuit 
was designed by a designer using schematic capture tools.
However, the partitioning task may be exceedingly difficult when its 
subject is a large netlist, e.g., generated automatically by a 
high--level synthesis tool.
An automatic partitioning tool is the only practical alternative in the 
latter case.

\bibliography{fpga}

\end{document}
