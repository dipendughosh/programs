
\documentstyle[11pt]{article}
\textheight 9.0in
\textwidth 6.5in
\oddsidemargin 0.0in
\topmargin -0.5in
\setlength{\floatsep}{0.3cm}
\setlength{\textfloatsep}{0.3cm}
\setlength{\intextsep}{0.3cm}
\newcommand{\definedas} {\stackrel{\Delta} {=}}
% Include the macros for including postscript figures in latex - CSG
%          psfig  Package GENERAL PUBLIC LICENSE
%                         17 Nov 1988
% This file is part of the TeXPS Software Package.

% The TeXPS Software Package is distributed in the hope that it will be useful,
% but WITHOUT ANY WARRANTY.  No author or distributor
% accepts responsibility to anyone for the consequences of using it
% or for whether it serves any particular purpose or works at all,
% unless he says so in writing.  Refer to the TeXPS Software Package
% General Public License for full details.

% Everyone is granted permission to copy, modify and redistribute
% the TeXPS Software Package, but only under the conditions described in the
% TeXPS Software Package General Public License.   A copy of this license is
% supposed to have been given to you along with TeXPS Software Package so you
% can know your rights and responsibilities.  It should be in a
% file named CopyrightLong.  Among other things, the copyright notice
% and this notice must be preserved on all copies.  */

\catcode`\@=11
%
% Changes done by Stephan Bechtolsheim:
% 1. Comments and indentation improved.
% 2. The \special which instructs the driver, is specified here.
% 3. Instead of numerical counter and dimension registers
%    symbolic register names are used.
% 4. scale parameter of \psfig introduced.
%
\def\@SpecialCodeDriver{dvitps: }
%
\newcount\psfc@a
\newcount\psfc@b
\newcount\psfc@c
\newcount\psfc@d
\newcount\psfc@e
\newcount\psfc@f
\newcount\psfc@g
\newcount\psfc@h
\newcount\psfc@i
\newcount\psfc@j
\newcount\psscale@count
\newtoks\psfigtoks@
% A dimension register for temporarily storing a dimension in
% \@pDimenToSpNumber.
\newdimen\psfig@dimen
%
\newwrite\@unused
\def\typeout#1{{\let\protect\string\immediate\write\@unused{#1}}}
% Identifying message is here.
\typeout{psfig/tex 1.4 / TeXPS}
%
% @psdo control structure -- similar to Latex @for.
% I redefined these with different names so that psfig can
% be used with TeX as well as LaTeX, and so that it will not 
% be vunerable to future changes in LaTeX's internal
% control structure.
\def\@nnil{\@nil}
\def\@empty{}
\def\@psdonoop#1\@@#2#3{}
\def\@psdo#1:=#2\do#3{\edef\@psdotmp{#2}\ifx\@psdotmp\@empty \else
    \expandafter\@psdoloop#2,\@nil,\@nil\@@#1{#3}\fi}
\def\@psdoloop#1,#2,#3\@@#4#5{\def#4{#1}\ifx #4\@nnil \else
       #5\def#4{#2}\ifx #4\@nnil \else#5\@ipsdoloop #3\@@#4{#5}\fi\fi}
\def\@ipsdoloop#1,#2\@@#3#4{\def#3{#1}\ifx #3\@nnil 
       \let\@nextwhile=\@psdonoop \else
      #4\relax\let\@nextwhile=\@ipsdoloop\fi\@nextwhile#2\@@#3{#4}}
\def\@tpsdo#1:=#2\do#3{\xdef\@psdotmp{#2}\ifx\@psdotmp\@empty \else
    \@tpsdoloop#2\@nil\@nil\@@#1{#3}\fi}
\def\@tpsdoloop#1#2\@@#3#4{\def#3{#1}\ifx #3\@nnil 
       \let\@nextwhile=\@psdonoop \else
      #4\relax\let\@nextwhile=\@tpsdoloop\fi\@nextwhile#2\@@#3{#4}}
% 
\def\psdraft{%
	\def\@psdraft{0}
	%\typeout{draft level now is \@psdraft \space.}
}
\def\psfull{%
	\def\@psdraft{100}
	%\typeout{draft level now is \@psdraft \space.}
}
\psfull
\newif\if@prologfile
\newif\if@postlogfile
\newif\if@noisy
\def\pssilent{%
	\@noisyfalse
}
\def\psnoisy{%
	\@noisytrue
}
\psnoisy
% These are for the option list: 
%	a specification of the form a = b maps to calling \@p@@sa{b}.
\newif\if@bbllx
\newif\if@bblly
\newif\if@bburx
\newif\if@bbury
\newif\if@height
\newif\if@width
\newif\if@rheight
\newif\if@rwidth
\newif\if@clip
\newif\if@scale
\newif\if@verbose
\def\@p@@sclip#1{\@cliptrue}
\def\@p@@sfile#1{% 
	%\typeout{file is #1}
	\def\@p@sfile{#1}% 
}
\def\@p@@sfigure#1{% 
	\def\@p@sfile{#1}% 
}
% \@pDimenToSpNumber
% ==================
% Convert a dimension into scaled points.
% #1: the name of macro which will expand to the dimension in
% 		scaled points, without the unit 'sp' though, i.e. as a pure
% 		integer.
% #2: the dimension (not a dimension register, use
% 		\the if dimension is stored in a dimension register).
\def\@pDimenToSpNumber #1#2{% 
	\psfig@dimen = #2\relax
	\edef#1{\number\psfig@dimen}%
}
\def\@p@@sbbllx#1{% 
	%\typeout{bbllx is #1}
	\@bbllxtrue
	\@pDimenToSpNumber{\@p@sbbllx}{#1}% 
}
\def\@p@@sbblly#1{% 
	%\typeout{bblly is #1}
	\@bbllytrue
	\@pDimenToSpNumber{\@p@sbblly}{#1}% 
}
\def\@p@@sbburx#1{%
	%\typeout{bburx is #1}
	\@bburxtrue
	\@pDimenToSpNumber{\@p@sbburx}{#1}% 
}
\def\@p@@sbbury#1{%
	%\typeout{bbury is #1}
	\@bburytrue
	\@pDimenToSpNumber{\@p@sbbury}{#1}% 
}
\def\@p@@sheight#1{
	\@heighttrue
	\@pDimenToSpNumber{\@p@sheight}{#1}% 
	%\typeout{Height is \@p@sheight}
}
\def\@p@@swidth#1{%
	%\typeout{Width is #1}
	\@widthtrue
	\@pDimenToSpNumber{\@p@swidth}{#1}% 
}
\def\@p@@srheight#1{
	%\typeout{Reserved height is #1}
	\@rheighttrue
	\@pDimenToSpNumber{\@p@srheight}{#1}% 
}
\def\@p@@srwidth#1{
	%\typeout{Reserved width is #1}
	\@rwidthtrue
	\@pDimenToSpNumber{\@p@srwidth}{#1}% 
}
\def\@p@@ssilent#1{% 
	\@verbosefalse
}
\def\@p@@sscale #1{% 
	\def\@p@scale{#1}%
	\@scaletrue
}

\def\@p@@sprolog#1{\@prologfiletrue\def\@prologfileval{#1}}
\def\@p@@spostlog#1{\@postlogfiletrue\def\@postlogfileval{#1}}
\def\@cs@name#1{\csname #1\endcsname}
\def\@setparms#1=#2,{\@cs@name{@p@@s#1}{#2}}
%
% Initialize the defaults.
%
\def\ps@init@parms{
	\@bbllxfalse \@bbllyfalse
	\@bburxfalse \@bburyfalse
	\@heightfalse \@widthfalse
	\@rheightfalse \@rwidthfalse
	\@scalefalse
	\def\@p@sbbllx{}\def\@p@sbblly{}
	\def\@p@sbburx{}\def\@p@sbbury{}
	\def\@p@sheight{}\def\@p@swidth{}
	\def\@p@srheight{}\def\@p@srwidth{}
	\def\@p@sfile{}
	\def\@p@scost{10}
	\def\@sc{}
	\@prologfilefalse
	\@postlogfilefalse
	\@clipfalse
	\if@noisy
		\@verbosetrue
	\else
		\@verbosefalse
	\fi
}
%
% Go through the options setting things up.
%
\def\parse@ps@parms#1{
 	\@psdo\@psfiga:=#1\do
	   {\expandafter\@setparms\@psfiga,}% 
}
%
% Compute bb height and width
%
\newif\ifno@bb
\newif\ifnot@eof
\newread\ps@stream
\def\bb@missing{%
	\if@verbose
		\typeout{psfig: searching \@p@sfile \space  for bounding box}% 
	\fi
	\openin\ps@stream=\@p@sfile
	\no@bbtrue
	\not@eoftrue
	\catcode`\%=12
	\ifeof\ps@stream
		\errmessage{FATAL ERROR: cannot open \@p@sfile}
	\fi
	\loop
		\read\ps@stream to \line@in
		\global\psfigtoks@=\expandafter{\line@in}
		\ifeof\ps@stream \not@eoffalse \fi
		%\typeout{ looking at :: \the\psfigtoks@ }
		\@bbtest{\psfigtoks@}
		\if@bbmatch\not@eoffalse\expandafter\bb@cull\the\psfigtoks@\fi
	\ifnot@eof \repeat
	\catcode`\%=14
}	
\newif\if@bbmatch

% '% ' becomes a regular character for a very short time.
{ 
	\catcode`\%=12
	\gdef\@bbtest#1{\expandafter\@a@\the#1%%BoundingBox:\@bbtest\@a@}
	\global\long\def\@a@#1%%BoundingBox:#2#3\@a@{\ifx\@bbtest#2\@bbmatchfalse\else\@bbmatchtrue\fi}
}

\long\def\bb@cull#1 #2 #3 #4 #5 {
	\@pDimenToSpNumber{\@p@sbbllx}{#2bp}%
	\@pDimenToSpNumber{\@p@sbblly}{#3bp}%
	\@pDimenToSpNumber{\@p@sbburx}{#4bp}%
	\@pDimenToSpNumber{\@p@sbbury}{#5bp}%
	\no@bbfalse
}
% Compute \@bbw and \@bbh, the width and height of the
% bounding box.
\def\compute@bb{
	\no@bbfalse
	\if@bbllx \else \no@bbtrue \fi
	\if@bblly \else \no@bbtrue \fi
	\if@bburx \else \no@bbtrue \fi
	\if@bbury \else \no@bbtrue \fi
	\ifno@bb \bb@missing \fi
	\ifno@bb
		\errmessage{\string\compute@bb: FATAL ERROR: no bounding box
		supplied (in \string\psfig) or found (in PostScript file).}
	\fi
	% Now compute the size of the bounding box.
	\psfc@c=\@p@sbburx
	\psfc@b=\@p@sbbury
	\advance\psfc@c by -\@p@sbbllx
	\advance\psfc@b by -\@p@sbblly
	\edef\@bbw{\number\psfc@c}
	\edef\@bbh{\number\psfc@b}
	%\typeout{\string\compute@bb: bbh = \@bbh, bbw = \@bbw}
}
%
% \in@hundreds performs #1 * (#2 / #3) correct to the hundreds,
%		then leaves the result in \@result.
%
\def\in@hundreds #1#2#3{% 
	\psfc@g=#2
	\psfc@d=#3
	\psfc@a=\psfc@g	% First digit #2/#3.
	\divide\psfc@a by \psfc@d
	\psfc@f=\psfc@a
	\multiply\psfc@f by \psfc@d
	\advance\psfc@g by -\psfc@f
	\multiply\psfc@g by 10
	\psfc@f=\psfc@g	% Second digit of #2/#3.
	\divide\psfc@f by \psfc@d
	\psfc@j=\psfc@f
	\multiply\psfc@j by \psfc@d
	\advance\psfc@g by -\psfc@j
	\multiply\psfc@g by 10
	\psfc@j=\psfc@g	% Third digit.
	\divide\psfc@j by \psfc@d
	\psfc@h=#1\psfc@i=0
	\psfc@e=\psfc@h
	\multiply\psfc@e by \psfc@a
	\advance\psfc@i by \psfc@e
	\psfc@e=\psfc@h
	\divide\psfc@e by 10
	\multiply\psfc@e by \psfc@f
	\advance\psfc@i by \psfc@e
	%
	\psfc@e=\psfc@h
	\divide\psfc@e by 100
	\multiply\psfc@e by \psfc@j
	\advance\psfc@i by \psfc@e
	%
	\edef\@result{\number\psfc@i}
}
% Scale a value #1 by the current scaling factor and reassign the new
% scaled value.
\def\@ScaleInHundreds #1{% 
	\in@hundreds{#1}{\@p@scale}{100}%
	\edef#1{\@result}% 
}
% 
%
% Compute width from height.
\def\compute@wfromh{
	% computing : width = height * (bbw / bbh)
	\in@hundreds{\@p@sheight}{\@bbw}{\@bbh}
	%\typeout{ \@p@sheight * \@bbw / \@bbh, = \@result }
	\edef\@p@swidth{\@result}
	%\typeout{w from h: width is \@p@swidth}
}
% Compute height from width.
\def\compute@hfromw{
	% computing : height = width * (bbh / bbw)
	\in@hundreds{\@p@swidth}{\@bbh}{\@bbw}
	%\typeout{ \@p@swidth * \@bbh / \@bbw = \@result }
	\edef\@p@sheight{\@result}
	%\typeout{h from w : height is \@p@sheight}
}
% Compute height and width, i.e. \@p@sheight and \@p@swidth.
\def\compute@handw{
	% If height is given.
	\if@height 
		% If width is given
		\if@width
		\else
			% Height, no width: compute width.
			\compute@wfromh
		\fi
	\else 
		% No height.
		\if@width
			% Width is given, no height though: compute it.
			\compute@hfromw
		\else
			% Neither width no height is give.
			\edef\@p@sheight{\@bbh}
			\edef\@p@swidth{\@bbw}
		\fi
	\fi
}
% Compute the amount of space to reserve. Unless defined
% using rheight and rwidth when \psfig is called, these values
% default to \@p@sheight and \@p@swidth.
\def\compute@resv{
	\if@rheight \else \edef\@p@srheight{\@p@sheight} \fi
	\if@rwidth \else \edef\@p@srwidth{\@p@swidth} \fi
}
%
%
% \psfig
% ======
% usage: \psfig{file=, height=, width=, bbllx=, bblly=, bburx=, bbury=,
%			rheight=, rwidth=, clip=, scale=}
%
% "clip=" is a switch and takes no value, but the `=' must be present.
\def\psfig#1{% 
	\vbox {%
		\ps@init@parms
		\parse@ps@parms{#1}
		% Compute any missing sizes.
		\compute@bb
		\compute@handw
		\compute@resv
		\if@scale
			\if@verbose
				\typeout{psfig: scaling by \@p@scale}% 
			\fi
			% We now scale the width and height as reported to the PS printer.
			\@ScaleInHundreds{\@p@swidth}% 
			\@ScaleInHundreds{\@p@sheight}% 
			\@ScaleInHundreds{\@p@srwidth}% 
			\@ScaleInHundreds{\@p@srheight}% 
		\fi
		%
		\ifnum\@p@scost<\@psdraft
			\if@verbose
				\typeout{psfig: including \@p@sfile \space}
			\fi
			% Cause "psfig.pro" to be included during pass0 processing of
			% the driver.
			\special{\@SpecialCodeDriver Include0 "psfig.pro"}
			% Now generate the instruction to include the figure.
			\special{% 
				\@SpecialCodeDriver Literal
				"\@p@swidth \space
				\@p@sheight \space
				\@p@sbbllx \space
				\@p@sbblly \space
				\@p@sbburx \space
				\@p@sbbury \space
				startTexFig \space"}
			% If clipping you may generate a message now.
			\if@clip
				\if@verbose
					\typeout{(clip)}
				\fi
				\special{\@SpecialCodeDriver Literal "doclip \space"}
			\fi
			\if@prologfile
			    \special{\@SpecialCodeDriver Include0 "\@prologfileval"}
			\fi
			\special{\@SpecialCodeDriver Include1 "\@p@sfile"}
			\if@postlogfile
			    \special{\@SpecialCodeDriver Include0 "\@postlogfileval"}
			\fi
			\special{\@SpecialCodeDriver Literal "endTexFig \space" }
			% End of \special generation.
			% Create a vbox to reserve the proper amount of space for the figure.
			\vbox to \@p@srheight true sp{
				\hbox to \@p@srwidth true sp{}
				\vss
			}
		\else
			% Draft modus: reserve the space for the figure and print the
			% path name.
			\vbox to \@p@srheight true sp{
				\hbox to \@p@srwidth true sp{%
					\if@verbose
						\@p@sfile
					\fi
				}
				\vss
			}
		\fi
	}
}

\catcode`\@=12



%\parindent 0pt
\begin{document}
% \setlength{\parindent}{5pt}
% \setlength{\parskip}{0.25cm}
\bibliographystyle{unsrt}

\begin{titlepage}
\vspace*{0.625in}

\begin{center}
{\Large \bf 
An Excerpt From: \\
\vskip 1.0cm
`Partitioning Digital Circuits   \\
\vskip 0.25cm
for Implementation in Multiple FPGA ICs' \\
}
\vspace{0.75cm}
{\normalsize \bf Roman Ku\v znar$^{1,2}$, Franc Brglez$^{1}$, Krzysztof
Kozminski$^{1}$} 
\vskip 1.0cm
\end{center}
\vspace{0.75in}
\footnotesize{

\indent The following pages contain an excerpt from the MCNC Technical Report
named above. This information is provided in order to facilitate the use of
the Partitioning (Xilinx) benchmarks developed as part of the project
described in this MCNC Technical Report.

\vspace{0.15in}
Memoranda in this series are issued for early dissemination to MCNC,
its six Participating Institutions, and MCNC Industrial Affiliates.
They are not considered to be published merely by virtue of appearing
in this series.  This copy is for private circulation only and may not
be further copied or distributed.

\vspace{0.15in}
MCNC Technical Reports are routinely submitted for publication
elsewhere.  Their distribution outside the MCNC Community prior to
publication is limited to peer communications and specific requests.
References to this work should be made either to the published version,
if any, or in the form ``private communication.''

\vspace{0.15in}
For information about the ideas expressed herein, contact the author(s)
directly.  For information about the MCNC Technical Report Series, or
Industrial Affiliates Program, contact Corporate Communications, MCNC,
P.O. Box 12889, Research Triangle Park, NC 27709; 919-248-1842.

\begin{center}

\vspace{0.25in}
\today

Technical Report TR93-03	
\vskip 0.3cm 
\copyright 1993 MCNC
All Rights Reserved
\end{center}
}

\vspace{0.875in}
\vspace{0.1in}
\footnoterule
\vskip 0.1in  
{\footnotesize 
\noindent
        $^{1}$MCNC Center for Microelectronic Systems Technologies,
        P.O. Box 12889,
        Research Triangle Park, NC 27709 \\

\noindent $^{2}$ Department of Electrical and Computer Engineering,
University of Ljubljana,
                 Tr\v za\v ska 25, 61000 Ljubljana, Slovenia
}
\end{titlepage}



%Cover_abstract.tex

\vskip 5.75cm

\begin{center}

{\Large \bf
Partitioning Digital Circuits   \\
\vskip 0.25cm
for Implementation in Multiple FPGA ICs \\
}
\vspace{0.75cm}
{\normalsize \bf Roman Ku\v znar, Franc Brglez, Krzysztof
Kozminski}

\vskip 3.0cm

{\large \bf ABSTRACT}
\end{center}
\vskip 0.5cm

This report considers the problem of obtaining a minimum--cost
partitioning of a large logic circuit into a collection of 
subcircuits such that each subcircuit will fit into some
device selected from a given library.
Each device in the library may have a different price, size, and
terminal capacity associated with it.
This work has been motivated by a need to provide digital system designers
with a capability to implement large circuits that do not fit in any one
of the commercially available FPGA devices.
Automatic partitioning software is needed to relieve the designers from the
tedious task of partitioning the design manually in such situations.

\vskip 6pt

The main result of the performed research is a partitioning algorithm
based on a recursive application of the Fiduccia-Mattheyses bipartitioning
heuristic.
This heuristic has been amended
with appropriate extensions to handle (a) the overall goal of the
cost minimization and (b) the constraints reflecting the
limitations on the capacity of FPGA chips.
While the formulation and the implementation 
are general, we demonstrate the application of our method
using circuits generated with a commercial tool and partitioned
for implementation with a specific library of Xilinx's FPGA devices.

\vskip 6pt

The experimental partitioning software, {\sl k-way.x}, developed in the
course of the reported work, has exhibited a very encouraging performance,
being able to produce solutions with cost quite near to the theoretical
minimum calculated for many of the benchmark circuits.




\section{Comments}
The following section contains a description of how the
{\em k--way.x} \cite{Kuznar93a} partitioning algorithm was interfaced to
the Xilinx tool, XACT. Other partitioning algorithms may be interfaced with
the Xilinx Design Development System in a similar manner. This document is
intended to provide users of the Partitioning benchmark suite with guidelines
about interfacing their partitioning algorithms to the Xilinx Design
Development System. 

\section{Software Implementation of the k--way Partitioning Algorithm}

We implemented the proposed k--way partitioning algorithm in the C 
language and designed interfaces to the Xilinx tool XACT \cite{Xil91a}.
In Figure \ref{k-way.interface} we illustrate the process of using our 
partitioning tool, {\sl k--way.x}, as an extension of the XACT Design 
Development System.

\subsection{Interface to the XACT Xilinx Development System}

\begin{figure}[htb]
\centerline{
\psfig{figure=fig.1,prolog=mac.pro}
}
%\vspace*{0.5in}
%\vspace*{6.2in}
\caption{Interfacing $k-way.x$ to XACT}
\label{k-way.interface}
\end{figure}



The input data to {\sl k--way.x} is a design description given in the 
$XNF$ format as a netlist of gate primitives.
Using the XACT software, this description can be readily mapped into a 
netlist of CLBs and IOBs represented in the $MAP$ format.
The need for partitioning arises when the mapped netlist does not fit 
into any FPGA package available in the device library.
Furthermore, partitioning can be used to minimize the design cost, even 
if it is possible to implement the design as a single, large FPGA IC.
For example, it is possible that the smallest FPGA chip with a 
sufficient number of CLBs is unroutable, forcing the designer to use
a larger device, capable of not only containing all CLBs but also 
yielding a 100\% routing completion.
This larger device may be significantly oversized compared to the 
design size; in other words, its CLB utilization may be very poor.
If the design is partitioned, it is possible that a number of smaller 
devices with higher CLB utilization could be used, thus reducing the 
design cost.
\par
Currently, the designer has to perform the partitioning manually.
In certain cases this task may be manageable, e.g., if the circuit 
was designed by a designer using schematic capture tools.
However, the partitioning task may be exceedingly difficult when its 
subject is a large netlist, e.g., generated automatically by a 
high--level synthesis tool.
An automatic partitioning tool is the only practical alternative in the 
latter case.

\bibliography{fpga}

\end{document}
