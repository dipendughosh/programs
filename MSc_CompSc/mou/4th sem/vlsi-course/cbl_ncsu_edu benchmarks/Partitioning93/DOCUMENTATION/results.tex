\documentstyle[11pt]{article}
\textheight 9.0in
\textwidth 6.5in
\oddsidemargin 0.0in
\topmargin -0.5in
\setlength{\floatsep}{0.3cm}
\setlength{\textfloatsep}{0.3cm}
\setlength{\intextsep}{0.3cm}
\newcommand{\definedas} {\stackrel{\Delta} {=}}


\begin{document}
\bibliographystyle{unsrt}

\begin{titlepage}
\vspace*{0.625in}

\begin{center}
{\Large \bf 
An Excerpt From: \\
\vskip 1.0cm
`Partitioning Digital Circuits   \\
\vskip 0.25cm
for Implementation in Multiple FPGA ICs' \\
}
\vspace{0.75cm}
{\normalsize \bf Roman Ku\v znar$^{1,2}$, Franc Brglez$^{1}$, Krzysztof
Kozminski$^{1}$} 
\vskip 1.0cm
\end{center}
\vspace{0.75in}
\footnotesize{

\indent The following pages contain an excerpt from the MCNC Technical Report
named above. This information is provided in order to facilitate the use of
the Partitioning (Xilinx) benchmarks developed as part of the project
described in this MCNC Technical Report.

\vspace{0.15in}
Memoranda in this series are issued for early dissemination to MCNC,
its six Participating Institutions, and MCNC Industrial Affiliates.
They are not considered to be published merely by virtue of appearing
in this series.  This copy is for private circulation only and may not
be further copied or distributed.

\vspace{0.15in}
MCNC Technical Reports are routinely submitted for publication
elsewhere.  Their distribution outside the MCNC Community prior to
publication is limited to peer communications and specific requests.
References to this work should be made either to the published version,
if any, or in the form ``private communication.''

\vspace{0.15in}
For information about the ideas expressed herein, contact the author(s)
directly.  For information about the MCNC Technical Report Series, or
Industrial Affiliates Program, contact Corporate Communications, MCNC,
P.O. Box 12889, Research Triangle Park, NC 27709; 919-248-1842.

\begin{center}

\vspace{0.25in}
\today

Technical Report TR93-03	
\vskip 0.3cm 
\copyright 1993 MCNC
All Rights Reserved
\end{center}
}

\vspace{0.875in}
\vspace{0.1in}
\footnoterule
\vskip 0.1in  
{\footnotesize 
\noindent
        $^{1}$MCNC Center for Microelectronic Systems Technologies,
        P.O. Box 12889,
        Research Triangle Park, NC 27709 \\

\noindent $^{2}$ Department of Electrical and Computer Engineering,
University of Ljubljana,
                 Tr\v za\v ska 25, 61000 Ljubljana, Slovenia
}
\end{titlepage}



%Cover_abstract.tex

\vskip 5.75cm

\begin{center}

{\Large \bf
Partitioning Digital Circuits   \\
\vskip 0.25cm
for Implementation in Multiple FPGA ICs \\
}
\vspace{0.75cm}
{\normalsize \bf Roman Ku\v znar, Franc Brglez, Krzysztof
Kozminski}

\vskip 3.0cm

{\large \bf ABSTRACT}
\end{center}
\vskip 0.5cm

This report considers the problem of obtaining a minimum--cost
partitioning of a large logic circuit into a collection of 
subcircuits such that each subcircuit will fit into some
device selected from a given library.
Each device in the library may have a different price, size, and
terminal capacity associated with it.
This work has been motivated by a need to provide digital system designers
with a capability to implement large circuits that do not fit in any one
of the commercially available FPGA devices.
Automatic partitioning software is needed to relieve the designers from the
tedious task of partitioning the design manually in such situations.

\vskip 6pt

The main result of the performed research is a partitioning algorithm
based on a recursive application of the Fiduccia-Mattheyses bipartitioning
heuristic.
This heuristic has been amended
with appropriate extensions to handle (a) the overall goal of the
cost minimization and (b) the constraints reflecting the
limitations on the capacity of FPGA chips.
While the formulation and the implementation 
are general, we demonstrate the application of our method
using circuits generated with a commercial tool and partitioned
for implementation with a specific library of Xilinx's FPGA devices.

\vskip 6pt

The experimental partitioning software, {\sl k-way.x}, developed in the
course of the reported work, has exhibited a very encouraging performance,
being able to produce solutions with cost quite near to the theoretical
minimum calculated for many of the benchmark circuits.




\section{Comments}
The following section describes the results obtained while using the 
{\em k--way.x} partioning algorithm \cite{Kuznar93a} to partition circuits
from the Partitioning benchmark suite. This document is intended to provide
users of the Partitioning benchmark suite with guidelines about reporting
results obtained using their partitioning algorithms on benchmark circuits,
and also some sample results.
  

\section{Experimental Results}

\label{sec-results}

Since we wanted to exercise our algorithm on a wide spectrum of 
different circuit sizes, and the commonly used layout synthesis 
benchmarks include only a few netlists with known functionality, we 
selected for our experiments a subset of the well--known test generation 
benchmarks \cite{iscas85,iscas89}.
As these circuits include full functional descriptions of all elements, 
we were able to map them into the XC2000 and XC3000 device families.
The characteristics of the benchmark circuits after mapping them into 
their respective families are shown in Table \ref{tb:tb2}.
A number of the benchmarks we used are clearly too large to be mapped 
into a single Xilinx's FPGA device, and even if better mappings could be 
found with somewhat lower CLB counts, the designer still would be faced 
with a partitioning problem to solve.
In any case, in the context of this report it is not important whether 
the mappings performed by the XACT tool are the best possible.
In this section, we discuss our experiences gained from using the {\sl 
k--way.x} tool on the chosen benchmarks, and demonstrate this program's 
effectiveness in minimizing the overall partitioning cost.

\subsection{Partitioning Experiments}

We used {\em k--way.x} in two partitioning experiments.
The benchmarks mapped into the XC2000 FPGA family were partitioned 
with the objective of obtaining an implementation using identical FPGA 
devices.
In this case, the problem is effectively simplified into partitioning 
into a smallest number of subsets, with constraints on the size and 
terminal count of each subset.
The benchmarks mapped into the XC3000 FPGA family were partitioned with 
the objective of obtaining a least--cost implementation using any 
variety of FPGA devices listed in Table 1. 
\par
For the circuits mapped into the XC2000 family, the numbers of CLBs 
range from 74 to 833 while the numbers of IOBs range from 64 to 313.
Considering the XC2064 device with 64 CLBs and 58 IOBs, the task is to 
partition the netlist into minimum number of devices, while meeting CLB 
and IOB constraints of the device.
As mentioned earlier, we also need to measure routability of the devices 
after partitioning.
The results of partitioning shown in Table \ref{tb:tb3} are encouraging
in several ways.
For 5 out of 8 circuits, we achieved the minimum cost possible, 
indicating that our implementation of the bisection algorithm is 
relatively stable and generates results which are consistent over a wide 
range of circuits.
For the remaining 3 circuits, we had to increase the number of 
subcircuits mostly due to the unusually large ratio of IOBs to CLBs in 
the original circuit, which had to be distributed between the individual 
subcircuits.
While not shown explicitly, only one subcircuit in one of the benchmarks 
was found initially unroutable and required backtracking with lower 
utilization bound to achieve 100\% routability.
The fact that none of these circuits had global clock or similar signals 
accounts for relatively high CLB utilization without a major impact on 
routability.
Finally, the CPU cost of partitioning is negligible compared to all 
other tasks, such as mapping, placement and routing performed by the 
XACT tool.
\par
For the circuits mapped into the XC3000 family, CLBs range from 283 to 
2904 while IOBs range from 64 to 313.
Considering the XC3000 device library in Table 1 with CLBs
ranging from 
64 to 320, IOB ranging from 64 to 144 and the relative price range from 
N\$1.00 to N\$4.83, the task of partitioning is not necessarily defined 
as finding a partition with the minimal number of devices but rather it 
is to minimize the overall cost of the partition.
Again, the results shown in Table \ref{tb:tb7} are encouraging.
We report results using the integer programming tool {\em lp--solve} to 
generate a lower bound for the lowest cost solution, followed by actual 
solutions generated with {\em k--way.x} in a static as well as a dynamic 
mode.
Partitions generated during the static mode of {\em k--way.x} basically 
attempt to match the device distribution provided by the {\em lp--solve} 
for the $initial$ circuit.
In contrast, subcircuits generated during the dynamic mode of
{\em k--way.x} always match a partition to the largest device returned by
{\em lp--solve} for the remainder subcircuit before the bisection.
The details of the dynamic version of the {\em k--way} algorithm have 
been given earlier.
\par
While it may be unrealistic to expect that we can meet the lower bound 
solutions for general circuits, both static and dynamic solutions in 
Table \ref{tb:tb7} appear reasonable, compared to the lower bound solutions
which have no constraints on the terminals or the number of cut nets allowed 
in the partition.
In both modes, the CPU costs are comparable.

\begin{table}[hpbt]
\begin{center}
\begin{tabular}{|c|c|c|c|c|c|c|c|}\hline
Device       &Type No. &$c_i$ &$t_i$ &$d_i$ &$l_i$  &$u_i$  &CLB cost\\ 
\hfill       &(i)      &(CLB) &(IOB) &(N\$) &\hfill &\hfill &$d_i / c_i$\\
\hline
\hline 
XC3020xx-xx  &1        &64    &64    &1.00  &0.8    &0.9    &0.0156\\
\hline
XC3030xx-xx  &2        &100   &80    &1.36  &0.8    &0.9    &0.0136\\
\hline
XC3042xx-xx  &3        &144   &96    &1.84  &0.8    &0.9    &0.0128\\
\hline
XC3064xx-xx  &4        &224   &110   &3.15  &0.8    &0.9    &0.0141\\
\hline
XC3090xx-xx  &5        &320   &144   &4.83  &0.8    &0.9    &0.0151\\
\hline 
\end{tabular}
\caption{A subset of the Xilinx XC3000 device library}
\label{tb:tb1}
\end{center}
\end{table}



\begin{table}[hpbt]
\begin{center}
\begin{tabular}{|c|c|c|c|c|c||c|}\hline
 Circuit   &\#CLBs  &\#IOBs  &\#DFF  &\#NETs  &\#PINs &\hfill  \\  \cline{1-6} \cline{1-6}
 c499      &74      &73      & -    &123     &409     &\hfill  \\ \cline{1-6}
 c1355     &74      &73      & -    &123     &408     &\hfill  \\ \cline{1-6}
 c1908     &147     &58      & -    &238     &780     &Mapped   \\ \cline{1-6}
 c2670     &210     &221     & -    &450     &1255    &into     \\ \cline{1-6}
 c3540     &373     &72      & -    &569     &1933    &XC2000   \\ \cline{1-6}
 c5315     &535     &301     & -    &936     &3004    &\hfill  \\ \cline{1-6}
 c6288     &833     &64      & -    &1472    &3438    &\hfill  \\ \cline{1-6}
 c7552     &611     &313     & -    &1057     &3318   &\hfill  \\ \hline \hline

 c3540     &283     &72      & -    &489     &1645    &\hfill  \\ \cline{1-6} 
 c5315     &377     &301     & -    &699     &2409    &\hfill \\ \cline{1-6}
 c6288     &833     &64      & -    &1472    &3438    &\hfill \\ \cline{1-6}
 c7552     &489     &313     & -    &921     &2924    &Mapped  \\ \cline{1-6}
 s5378     &381     &86      &179   &628     &2332    &into    \\ \cline{1-6}
 s9234     &454     &43      &288   &716     &2671    &XC3000  \\ \cline{1-6}
 s13207    &915     &154     &661   &1377    &5309    &\hfill  \\ \cline{1-6}
 s15850    &842     &102     &597   &1265    &4977    &\hfill    \\ \cline{1-6}
 s38584    &2904    &292     &1452  &3884    &17483   &\hfill \\ \hline
\end{tabular}
\caption{Benchmark circuit characteristics}
\label{tb:tb2}
\end{center}
\end{table}



\begin{table}[hpbt]
\begin{center}
\begin{tabular}{|c|c|c|c|c|c|c|c|c|c|}\hline
 Circuit   &Cost         &Cost             &CPU      &\multicolumn{3}{c|}{CLB}          &\multicolumn{3}{c|}{IOB} \\  
 \hfill    &(Lower bound)&(achieved)       &Cost     &\multicolumn{3}{c|}{utilization}  &\multicolumn{3}{c|}{utilization}\\ \cline{5-10}
\hfill     &(N\$)         &(N\$)             &(sec.)   &min.    &avg.    &max.            &min.     &avg.    &max. \\ \hline \hline
 c499      &2.00            &2.00            &4.2      &0.47    &0.56    &0.69            &0.95     &0.96    &0.97 \\ \hline 
 c1355     &2.00            &2.00            &4.4      &0.58    &0.58    &0.58            &0.95     &0.95    &0.95 \\ \hline    
 c1908     &3.00            &3.00            &15.9     &0.64    &0.77    &0.89            &0.64     &0.79    &0.98 \\ \hline 
 c2670     &4.00            &6.00            &51.6     &0.28    &0.55    &0.92            &0.74     &0.91    &1.00 \\ \hline 
 c3540     &6.00            &6.00            &59.4     &0.94    &0.97    &1.00            &0.74     &0.93    &1.00 \\ \hline
 c5315     &9.00            &11.00           &176.8    &0.55    &0.77    &0.89            &0.78     &0.91    &1.00 \\ \hline
 c6288     &14.00           &14.00           &306.2    &0.84    &0.94    &1.00            &0.43     &0.57    &0.69 \\ \hline 
 c7552     &10.00           &11.00           &239.1    &0.48    &0.88    &1.00            &0.72     &0.90    &1.00 \\ \hline
\end{tabular}
\caption{Partitioning using XC2000 devices of a single type (\$1.00/device)}
\label{tb:tb3}
\end{center}
\end{table}



\begin{table}[hpbt]
\begin{center}
\begin{tabular}{|c|c|c|c|c|}\hline
 Circuit      &Device         &Total Cost   &CPU Cost  &\hfill  \\
 \hfill       &Distribution   &(N\$)        &(sec.)    &\hfill \\ \cline{1-4}\cline{1-4}
 c3540        &\{0,1,0,1,0\}  &4.51         & 0.1      &\hfill \\ \cline{1-4}
 c5315        &\{0,3,1,0,0\}  &5.92         & 0.1      &\hfill \\ \cline{1-4}
 c6288        &\{0,5,3,0,0\}  &12.33        & 0.1      &Lower bound  \\ \cline{1-4}
 c7552        &\{0,0,4,0,0\}  &7.36         & 0.1      &solutions   \\ \cline{1-4}
 s5378        &\{0,0,3,0,0\}  &5.52         & 0.1      &using  \\ \cline{1-4}
 s9234        &\{0,0,2,1,0\}  &6.83         & 0.1      &{\sl lp-solve} \\ \cline{1-4}
 s13207       &\{1,1,6,0,0\}  &13.40        & 0.1      &\hfill \\ \cline{1-4}
 s15850       &\{0,0,5,1,0\}  &12.35        & 0.1      &\hfill \\ \cline{1-4}
 s38548       &\{0,1,0,14,0\} &45.50        & 0.1      &\hfill \\ \hline
\end{tabular}
\end{center}

\vspace*{5mm}
\begin{center}
\begin{tabular}{|c|c|c|c|c|c|c|}\hline
 Circuit      &Device         &Total Cost   &CPU Cost  &CLB           &IOB         & \hfill \\
 \hfill       &Distribution   & (N\$)        &(sec.)    &Utilization   &Utilization & \hfill \\
\cline{1-6} \cline{1-6}

 c3540        &\{1,1,0,1,0\}  &5.51         & 26.2     &0.74          &0.82        & \hfill \\ \cline{1-6}
 c5315        &\{0,0,5,0,0\}  &9.19         & 77.8     &0.52          &0.95        & \hfill \\ \cline{1-6}
 c6288        &\{0,6,3,0,0\}  &13.68        & 88.6     &0.80          &0.50        & Static \\ \cline{1-6}
 c7552        &\{0,0,4,0,0\}  &7.36         & 71.1     &0.85          &0.93        & solutions \\ \cline{1-6}
 s5378        &\{0,1,4,0,0\}  &8.72         & 69.7     &0.56          &0.92        & using \\ \cline{1-6}
 s9234        &\{0,0,3,1,0\}  &8.68         & 87.8     &0.69          &0.74        & {\sl k-way.x} \\ \cline{1-6}
 s13207       &\{9,1,6,0,0\}  &21.39        & 227.5    &0.59          &0.92        & \hfill \\ \cline{1-6}
 s15850       &\{0,1,6,1,0\}  &15.59        & 206.5    &0.71          &0.79        & \hfill \\ \cline{1-6}
 s38584       &\{3,6,0,13,0\} &52.15        & 493.4    &0.78          &0.43        & \hfill \\ \hline
\end{tabular}
\end{center}
 
\vspace*{5mm}
\begin{center}
\begin{tabular}{|c|c|c|c|c|c|c|}\hline
 Circuit      &Device         &Total Cost   &CPU Cost  &CLB           &IOB         &\hfill \\
 \hfill       &Distribution   & (N\$)        &(sec.)    &Utilization   &Utilization &\hfill \\ \cline{1-6} \cline{1-6}
 
 c3540        &\{0,0,3,0,0\}  &5.52         & 26.2     &0.86          &0.73        &\hfill \\ \cline{1-6}
 c5315        &\{2,1,2,0,0\}  &7.03         & 79.4     &0.73          &0.89        &\hfill \\ \cline{1-6}
 c6288        &\{0,0,4,2,0\}  &13.66        & 78.6     &0.96          &0.50        &Dynamic \\ \cline{1-6}
 c7552        &\{0,0,4,0,0\}  &7.36         & 73.4     &0.85          &0.93        &solutions \\ \cline{1-6}
 s5378        &\{0,0,1,0,1\}  &6.67         & 68.4     &0.82          &0.82        &using    \\
\cline{1-6}
 s9234        &\{0,0,0,1,1\}  &7.98         & 63.3     &0.77          &0.76        &{\sl k-way.x}  \\ \cline{1-6}
 s13207       &\{3,5,4,0,0\}  &17.16        & 239.8    &0.72          &0.88        &\hfill \\ \cline{1-6}
 s15850       &\{0,0,2,2,1\}  &14.80        & 171.8    &0.83          &0.78        &\hfill \\ \cline{1-6}
 s38584       &\{0,5,15,4,1\} &51.83        & 466.5    &0.75          &0.67        &\hfill \\ \hline
\end{tabular}
\caption{
Partitioning using XC3000 device library in Table 1
}
\label{tb:tb7}
\end{center}
\end{table}


\bibliography{fpga}


\end{document}
